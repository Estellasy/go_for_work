% !TeX TS-program = xelatex

\documentclass{resume}
\ResumeName{任思怡}

% 如果想插入照片,请使用以下两个库。
% \usepackage{graphicx}
% \usepackage{tikz}

\begin{document}

\ResumeContacts{
  (+86)173-9612-6620,%
  \ResumeUrl{mailto:siyiren1@foxmail.com}{siyiren1@foxmail.com},%
  % \ResumeUrl{https://blog.fkynjyq.com}{blog.fkynjyq.com} \footnote{下划线内容包含超链接。},%
  % \ResumeUrl{https://github.com/fky2015}{github.com/fky2015}%
  wechat: cosmoos,
  意向: 软件开发-golang方向(实习)
}

% 如果想插入照片,请取消此代码的注释。
% 但是默认不推荐插入照片,因为这不是简历的重点。
% 如果默认的照片插入格式不能满足你的需求,你可以尝试调整照片的大小,或者使用其他的插入照片的方法。
% 不然,也可以先渲染 PDF 简历,然后用其他工具在 PDF 上叠加照片。
% \begin{tikzpicture}[remember picture, overlay]
%   \node [anchor=north east, inner sep=1cm]  at (current page.north east) 
%      {\includegraphics[width=2cm]{image.png}};
% \end{tikzpicture}

\ResumeTitle


\section{教育经历}
\ResumeItem
[华中科技大学|硕士研究生]
{华中科技大学(985)}
[\textnormal{计算机科学与技术,计算机学院|}  学术型硕士研究生]
[2022.09—2025.06(预计)]
\textbf{GPA: 3.62/4.0}。主要研究方向为面向工业缺陷检测任务的半监督学习,一篇CCF-C论文在审。\newline
获校优秀研究生干部(2023),研究生数学建模竞赛三等奖(2022)。\textbf{2025年应届生}。

\ResumeItem
[华中师范大学|本科生]
{华中师范大学(211)}
[\textnormal{计算机科学与技术,计算机学院|} 工学学士]
[2018.09—2022.06]

\textbf{GPA: 3.6/4.0(专业前 10\%)}。主修课程包括数据结构、操作系统、计算机网络、算法设计与分析等。\newline
获校三好学生(3次)、校优秀毕业生(2022),全国大学生数学竞赛三等奖(2021)、全国大学生数学建模竞赛二等奖(2020)、蓝桥杯湖北省赛二等奖(2019)等。

\section[技术能力]{技术能力}
\begin{itemize}
  \item \textbf{编程语言}:熟悉Golang,了解GMP调度模型、GC原理、channel与内存管理相关知识。
  \item \textbf{脚本语言}:具有Python机器学习、深度学习、爬虫相关经验,熟悉Python脚本编写。
  \item \textbf{数据库}:熟悉MySQL的基本原理和使用,了解索引、事务、锁机制。了解Redis,了解内部数据结构、事务、内存淘汰策略等。
  % \item \textbf{数据结构和算法}:熟悉基本数据结构,如栈、队列、链表、树等。熟悉回溯、动态规划等算法。
  \item \textbf{计算机网络和操作系统}:了解OSI七层模型,熟悉TCP/IP、HTTP等常用网络协议。熟悉操作系统进程间通信、内存管理、文件管理相关知识。
  \item \textbf{工作流}:熟悉Linux常用命令,了解Git常用命令。
\end{itemize}

% \section{工作经历}

% \ResumeItem{北京 ABCD 有限公司}
% [后端开发实习生/XXXX]
% [2020.10—2021.03] 

% \begin{itemize}
%   \item \textbf{独立负责XXX业务后端的设计、开发、测试和部署。}通过 FaaS、Kafka 等平台实现站内信模板渲染服务。向上游提供 SDK 代码,增加或升级了多种离线和在线逻辑。完成了业务对站内信的多样需求。
%   \item \textbf{参与 XXX 的需求分析,系统技术方案设计;完成需求开发、灰度测试、上线和监控。}
% \end{itemize}

\section{项目经历}

\ResumeItem{\textbf{bitcask-go} | go语言实现的基于bitcask模型的高性能存储引擎} 
\begin{itemize}
    \item \textbf{项目描述}:bitcask 是一种基于预写日志的高性能持久化存储引擎,用于存储key-value数据。
    本项目使用 Go 语言实现了 Bitcask 存储模型,实现了快速的写入、读取和删除操作,最多只需一次磁盘 I/O。
    \item \textbf{持久化}:实现了数据持久化功能,确保数据的可靠性和可恢复性。实现了bitcask模型的merge方法,对旧WAL数据进行合并去重,清理无效数据,减少磁盘空间占用。
    \item \textbf{索引}:基于google/btree实现索引,实现高效、快速的数据访问,支持范围查找。
    \item \textbf{并发控制}:使用读写锁,确保数据的一致性和并发访问的正确性。
    \item \textbf{性能优化}:实现了bitcask模型的hint文件,采用WAL方式记录key和索引,避免重启时全量加载所有数据构建内存索引,提高启动速度。
    % \item \textbf{关键收获}:对kv数据库有了基本的认识和实践,对数据文件空间有了进一步理解。
\end{itemize}

\ResumeItem{\textbf{go-cloud-disk} | go语言实现的云存储系统}
\begin{itemize}
    \item \textbf{项目描述}:采用Go语言实现的云存储系统,整合了腾讯云COS和MinIO集群,提供了文件上传和下载服务。
    \item \textbf{用户鉴权}:整合了gomail发送验证码和第三方QQ身份验证功能,支持验证码登录、密码登录和扫码登录。利用JWT进行会话管理,确保安全性和用户体验。
    \item \textbf{文件传输}:实现了文件分块上传、断点续传和文件秒传功能。利用Goroutine和channel处理文件分块和合并问题,使用Redis记录传输状态。
    \item \textbf{性能优化}:基于批量消息处理思想,采用 Kafka、batcher 和 Goroutine 实现文件元信息的高并发处理。
    在生产端使用 batcher 进行消息聚合,提高吞吐性能;在消费端使用多个 Goroutine 进行并发消费。
    % \item \textbf{关键收获}:对分布式文件系统有了基本的认识和实践,对文件传输和存储有了进一步的理解。
\end{itemize}

\section{个人总结}
\begin{itemize}
  \item 乐观开朗,待人平和真诚,在校期间多次参加数学建模竞赛,解决实际问题和快速学习新技术的能力较强,具有良好的沟通能力和团队合作精神。
  \item 经常阅读英文论文和文档,英语六级成绩521,其中阅读200+。
  \item 涉猎广泛,积极尝试使用大语言模型等应用,并持续分享和探索这些应用的申请和使用方法。
\end{itemize}
\end{document}
